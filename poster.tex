%%%%%%%%%%%%%%%%%%%%%%%%%%%%%%%%%%%%%%%%%
% baposter Landscape Poster
% LaTeX Template
% Version 1.0 (11/06/13)
%
% baposter Class Created by:
% Brian Amberg (baposter@brian-amberg.de)
%
% This template has been downloaded from:
% http://www.LaTeXTemplates.com
%
% License:
% CC BY-NC-SA 3.0 (http://creativecommons.org/licenses/by-nc-sa/3.0/)
%
%%%%%%%%%%%%%%%%%%%%%%%%%%%%%%%%%%%%%%%%%

%----------------------------------------------------------------------------------------
%	PACKAGES AND OTHER DOCUMENT CONFIGURATIONS
%----------------------------------------------------------------------------------------

\documentclass[landscape,a0paper,fontscale=0.285]{baposter} % Adjust the font scale/size here

\usepackage{graphicx} % Required for including images
\graphicspath{{figures/}} % Directory in which figures are stored

\usepackage{amsmath} % For typesetting math
\usepackage{amssymb} % Adds new symbols to be used in math mode

\usepackage{booktabs} % Top and bottom rules for tables
\usepackage{enumitem} % Used to reduce itemize/enumerate spacing
\usepackage{palatino} % Use the Palatino font
\usepackage[font=small,labelfont=bf]{caption} % Required for specifying captions to tables and figures

\usepackage{multicol} % Required for multiple columns
\setlength{\columnsep}{1.5em} % Slightly increase the space between columns
\setlength{\columnseprule}{0mm} % No horizontal rule between columns

\usepackage{tikz} % Required for flow chart
\usetikzlibrary{shapes,arrows} % Tikz libraries required for the flow chart in the template

\newcommand{\compresslist}{ % Define a command to reduce spacing within itemize/enumerate environments, this is used right after \begin{itemize} or \begin{enumerate}
\setlength{\itemsep}{1pt}
\setlength{\parskip}{0pt}
\setlength{\parsep}{0pt}
}

\definecolor{lightblue}{rgb}{1,0,0} % Defines the color used for content box headers

\begin{document}

\begin{poster}
{
headerborder=closed, % Adds a border around the header of content boxes
colspacing=1em, % Column spacing
bgColorOne=white, % Background color for the gradient on the left side of the poster
bgColorTwo=white, % Background color for the gradient on the right side of the poster
borderColor=lightblue, % Border color
headerColorOne=black, % Background color for the header in the content boxes (left side)
headerColorTwo=lightblue, % Background color for the header in the content boxes (right side)
headerFontColor=white, % Text color for the header text in the content boxes
boxColorOne=white, % Background color of the content boxes
textborder=roundedleft, % Format of the border around content boxes, can be: none, bars, coils, triangles, rectangle, rounded, roundedsmall, roundedright or faded
eyecatcher=true, % Set to false for ignoring the left logo in the title and move the title left
headerheight=0.1\textheight, % Height of the header
headershape=roundedright, % Specify the rounded corner in the content box headers, can be: rectangle, small-rounded, roundedright, roundedleft or rounded
headerfont=\Large\bf\textsc, % Large, bold and sans serif font in the headers of content boxes
%textfont={\setlength{\parindent}{1.5em}}, % Uncomment for paragraph indentation
linewidth=2pt % Width of the border lines around content boxes
}
%----------------------------------------------------------------------------------------
%	TITLE SECTION 
%----------------------------------------------------------------------------------------
%
{\includegraphics[height=4em]{logo.png}} % First university/lab logo on the left
{\bf\textsc{<Title>}\vspace{0.5em}} % Poster title
{\textsc{\{ Mark Baranowski, Ian Briggs, Ganesh Gopalakrishnan\} \hspace{12pt} University of Utah School of Computing}} % Author names and institution
{\includegraphics[height=4em]{logo.png}} % Second university/lab logo on the right


%----------------------------------------------------------------------------------------
%	BoxOne
%----------------------------------------------------------------------------------------

\headerbox{<BoxOne>}{name=BoxOne,column=0,row=0}{

\vspace{2in} % remove once there is content

%\vspace{0.3em} % When there are two boxes, some whitespace may need to be added if the one on the right has more content
}





%----------------------------------------------------------------------------------------
%	BoxTwo
%----------------------------------------------------------------------------------------

\headerbox{<BoxTwo>}{name=BoxTwo,column=1,row=0}{

\vspace{2in} % remove once there is content

%\vspace{0.3em} % When there are two boxes, some whitespace may need to be added if the one on the right has more content
}





%----------------------------------------------------------------------------------------
%	BoxThree
%----------------------------------------------------------------------------------------

\headerbox{<BoxThree>}{name=BoxThree,column=2,span=2,row=0}{

\vspace{2in} % remove once there is content

}











%----------------------------------------------------------------------------------------
%	BoxFour
%----------------------------------------------------------------------------------------

\headerbox{<BoxFour>}{name=BoxFour,column=0,below=BoxOne}{ 

\vspace{2.5in} % remove once there is content

%\vspace{0.3em} % When there are two boxes, some whitespace may need to be added if the one on the right has more content
}




%----------------------------------------------------------------------------------------
%	BoxFive
%----------------------------------------------------------------------------------------

\headerbox{<BoxFive>}{name=BoxFive,column=1,below=BoxTwo}{ 

\vspace{2.5in} % remove once there is content

%\vspace{0.3em} % When there are two boxes, some whitespace may need to be added if the one on the right has more content
}




%----------------------------------------------------------------------------------------
%	BoxSix
%----------------------------------------------------------------------------------------

\headerbox{<BoxSix>}{name=BoxSix,column=2,span=2,row=0,below=BoxThree}{

\vspace{2.5in} % remove once there is content

}
















%----------------------------------------------------------------------------------------
%	BoxSeven
%----------------------------------------------------------------------------------------

\headerbox{<BoxSeven>}{name=BoxSeven,column=0,below=BoxFour,above=bottom}{

\vspace{2in} % remove once there is content

%\vspace{0.3em} % When there are two boxes, some whitespace may need to be added if the one on the right has more content
}



%----------------------------------------------------------------------------------------
%	BoxEight
%----------------------------------------------------------------------------------------

\headerbox{<BoxEight>}{name=BoxEight,column=1,below=BoxFive,above=bottom}{

\vspace{2in} % remove once there is content

%\vspace{0.3em} % When there are two boxes, some whitespace may need to be added if the one on the right has more content
}



%----------------------------------------------------------------------------------------
%	BoxNine
%----------------------------------------------------------------------------------------

\headerbox{<BoxNine>}{name=BoxNine,column=2,span=2,below=BoxSix,above=bottom}{

\vspace{2in} % remove once there is content

}







%----------------------------------------------------------------------------------------

\end{poster}

\end{document}