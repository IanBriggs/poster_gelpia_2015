%%%%%%%%%%%%%%%%%%%%%%%%%%%%%%%%%%%%%%%%%%%%%%%%%%%%%%%%%%%%%%%%%%%%%%%%%%%%%%%%%%%%%%%%%%%%%%%%%%%%%
% baposter Landscape Poster
% LaTeX Template
% Version 1.0 (11/06/13)
%
% baposter Class Created by:
% Brian Amberg (baposter@brian-amberg.de)
%
% This template has been downloaded from:
% http://www.LaTeXTemplates.com
%
% License:
% CC BY-NC-SA 3.0 (http://creativecommons.org/licenses/by-nc-sa/3.0/)
%
%%%%%%%%%%%%%%%%%%%%%%%%%%%%%%%%%%%%%%%%%%%%%%%%%%%%%%%%%%%%%%%%%%%%%%%%%%%%%%%%%%%%%%%%%%%%%%%%%%%%%

%----------------------------------------------------------------------------------------------------
%	PACKAGES AND OTHER DOCUMENT CONFIGURATIONS
%----------------------------------------------------------------------------------------------------

\documentclass[landscape,a0paper,fontscale=0.285]{baposter} % Adjust the font scale/size here

\usepackage{wrapfig}
\usepackage{graphicx} % Required for including images
\graphicspath{{figures/}} % Directory in which figures are stored

\usepackage{amsmath} % For typesetting math
\usepackage{amssymb} % Adds new symbols to be used in math mode

\usepackage{booktabs} % Top and bottom rules for tables
\usepackage{enumitem} % Used to reduce itemize/enumerate spacing
\usepackage{palatino} % Use the Palatino font
\usepackage[font=small,labelfont=bf]{caption} % Required for setting captions to tables and figures

\usepackage{multicol} % Required for multiple columns
\setlength{\columnsep}{1.5em} % Slightly increase the space between columns
\setlength{\columnseprule}{0mm} % No horizontal rule between columns

\usepackage{tikz} % Required for flow chart
\usetikzlibrary{shapes,arrows} % Tikz libraries required for the flow chart in the template

% Define a command to reduce spacing within itemize/enumerate environments,
% this is used right after \begin{itemize} or \begin{enumerate}
\newcommand{\compresslist}{
\setlength{\itemsep}{1pt}
\setlength{\parskip}{0pt}
\setlength{\parsep}{0pt}
}

\definecolor{lightblue}{rgb}{1,0,0} % Defines the color used for content box headers

\begin{document}

\begin{poster}
{
headerborder=closed, % Adds a border around the header of content boxes
colspacing=1em, % Column spacing
bgColorOne=white, % Background color for the gradient on the left side of the poster
bgColorTwo=white, % Background color for the gradient on the right side of the poster
borderColor=lightblue, % Border color
headerColorOne=black, % Background color for the header in the content boxes (left side)
headerColorTwo=lightblue, % Background color for the header in the content boxes (right side)
headerFontColor=white, % Text color for the header text in the content boxes
boxColorOne=white, % Background color of the content boxes
% Format of the border around content boxes,
% can be: none, bars, coils, triangles, rectangle, rounded, roundedsmall, roundedright or faded
textborder=roundedleft,
eyecatcher=true, % Set to false for ignoring the left logo in the title and move the title left
headerheight=0.1\textheight, % Height of the header
% Specify the rounded corner in the content box headers,
% can be: rectangle, small-rounded, roundedright, roundedleft or rounded
headershape=roundedright,
headerfont=\Large\bf\textsc, % Large, bold and sans serif font in the headers of content boxes
%textfont={\setlength{\parindent}{1.5em}}, % Uncomment for paragraph indentation
linewidth=2pt % Width of the border lines around content boxes
}
%----------------------------------------------------------------------------------------------------
%	TITLE SECTION
%----------------------------------------------------------------------------------------------------
{\includegraphics[height=4em]{left_logo}} % First university/lab logo on the left
{\bf\textsc{<Title>}\vspace{0.5em}} % Poster title
{\textsc{ {\large \{Mark Baranowski, Ian Briggs, Ganesh Gopalakrishnan, Zvonimir Rakamaric\}}
    \hspace{12pt} University of Utah School of Computing}} % Authors and institution
{\includegraphics[height=4em]{right_logo}} % Second university/lab logo on the right











%----------------------------------------------------------------------------------------------------
% Introduction/Motivation
%----------------------------------------------------------------------------------------------------
\headerbox{Motivation}{name=BoxOne,column=0}{
  % Intro/Motivation can be the importance of glob opt in many contexts including
  % FP error bounding and FP error bounding can be motivated via its relevance to
  % HPC code precision tuning

  Floating point computation is inherently flawed. Each operation carries with it an error term
  that, with many operations compounding, can cause tremendous error in the output. When using
  floating point arithmetic one must be aware of the magnatude of the error in the computation being
  done. This is where floating point error bounding comes into play. It can be used to give tight
  upper bounds to computations, such as those used in high powered computing. To power this bounding
  is global optimzation utilizing interval arithmatic.

  % When there are two boxes, some whitespace may need to be added if the
  % one on the right has more content
  % \vspace{0.3em}
}





%----------------------------------------------------------------------------------------------------
% Serial Algorithm
%----------------------------------------------------------------------------------------------------
\headerbox{Serial Algorithm}{name=BoxTwo, column=0, below=BoxOne}{
  The classical branch and bound algorithm utilizes the property of interval arithmatic that
  computation of a function over an interval gives an answer bound that is guaranteed to contain the
  real bounds of the function within the inputs given. A high water mark is used to store the highest
  lower bound seen, and if a given input produces an output which lies completely below this mark
  then the input can be discarded. Otherwise the input is split in half and this computation is
  repeated on each new input. An input will also be discarded if the maximum width is below a user
  set threshold. Once there are no more inputs to process the maximum of the function is tightly
  bounded by the highest upper output of the inputs which were discarded due to width tolerance.

  % When there are two boxes, some whitespace may need to be added if the
  % one on the right has more content
  % \vspace{0.3em}
}





%----------------------------------------------------------------------------------------------------
% Parallel Algorithms
%----------------------------------------------------------------------------------------------------
\headerbox{Parallel Algorithms}{name=BoxThree, column=0, span=2, below=BoxTwo, above=bottom}{
  There is considerable shared state in the traditional algorithm. The high water mark, the inputs
  which are still to be processed, and the maximum as it is found. To midigate the problems
  associated with shared state we experimented with various methods. One consisted of using parallel
  map functions on a list of the inputs to be process, each map produced a list of new inputs, a list
  of new high water marks, and a list of new solutions. The latter two were reduced to find the
  maximum and the process repeated. Another solver split the initial input into many pieces and sent
  those off to worker threads. When a worker thread finished work on it's input it would report back
  to the main thread with a new water mark and maximum, if found, and the main thread would in turn
  given the thread a new input to work on. If the new values were better than the main threads values
  an update would be issued to all worker threads to update their local values. 

  % When there are two boxes, some whitespace may need to be added if the
  % one on the right has more content
  % \vspace{0.3em}
}















\newlength{\graphWidth}
\setlength{\graphWidth}{3.6in}
%----------------------------------------------------------------------------------------------------
% Function complexity vs time graph
%----------------------------------------------------------------------------------------------------
\headerbox{Function Complexity}{name=BoxFour, column=1, span=2}{
  \begin{tabular}{p{.95\linewidth-\graphWidth}p{\graphWidth}}
    The number of input variables to a function causes an exponential increase in the search space
    and solving time. There are tricks to tame this complexity. Partitioning the function into
    expressions which contain no overlapping variables is one example. The problem with these
    simplifications is that they only apply to a subset of input functions. If the function cannot
    be split into expressions it must be solved as a whole.
    &
    \raisebox{-.9\totalheight}{\includegraphics[width=\graphWidth]{graph_one}}
  \end{tabular}
  % When there are two boxes, some whitespace may need to be added if the
  % one on the right has more content
  % \vspace{0.3em}
}




%----------------------------------------------------------------------------------------------------
% Loop iteration comparison of solvers
%----------------------------------------------------------------------------------------------------
\headerbox{Loop Iterations}{name=BoxFive, column=1, span=2, below=BoxFour, above=BoxThree}{
  \begin{tabular}{p{\graphWidth}p{.95\linewidth-\graphWidth}}
    \raisebox{-.9\totalheight}{\includegraphics[width=\graphWidth]{graph_two}}
    &
    Since there are multiple paths of inquery when the parallel solvers are running there is a
    difference in the number of total loops involved. This indicates how smart and efficent each
    search strategy is. This is a more reliable method of comparing solvers than wallclock time.
  \end{tabular}
  % When there are two boxes, some whitespace may need to be added if the
  % one on the right has more content
  % \vspace{0.3em}
}





%----------------------------------------------------------------------------------------------------
% Swig
%----------------------------------------------------------------------------------------------------
\headerbox{Swig}{name=BoxSix, column=2, below=BoxFive, above=bottom}{
  To utilize existing libraaries for performing interval arithmatic we used the MPFI library. To
  allow for quick prototyping we elected to use python as the language for the main algorithm. In
  order to interface these two langauge we used the Simplified Wrapper and Interface Generator, SWIG.
  This quick prototyping allowed leway for ideas to be quickly tested before much work went into
  them. Python was never ment to be the end target, just a way to query the field of ideas quickly
  so a suitable algorithm could be found for the final product.

  % When there are two boxes, some whitespace may need to be added if the
  % one on the right has more content
  % \vspace{0.3em}
}
















%----------------------------------------------------------------------------------------------------
% Hidden costs of python
%----------------------------------------------------------------------------------------------------
\headerbox{Python Oversights}{name=BoxSeven, column=3}{
  

  % When there are two boxes, some whitespace may need to be added if the
  % one on the right has more content
  % \vspace{0.3em}
}





%----------------------------------------------------------------------------------------------------
% Move to rust
%----------------------------------------------------------------------------------------------------
\headerbox{Move to rust}{name=BoxEight, column=3, below=BoxSeven}{


  % When there are two boxes, some whitespace may need to be added if the
  % one on the right has more content
  % \vspace{0.3em}
}



%----------------------------------------------------------------------------------------------------
% References
%----------------------------------------------------------------------------------------------------

\headerbox{References}{name=BoxNine, column=3, below=BoxEight}{

\renewcommand{\section}[2]{\vskip 0.05em} % Get rid of the default "References" section title
\nocite{*} % Insert publications even if they are not cited in the poster
\small{ % Reduce the font size in this block
\bibliographystyle{unsrt}
\bibliography{biblio} % Use sample.bib as the bibliography file
}

  % When there are two boxes, some whitespace may need to be added if the
  % one on the right has more content
  % \vspace{0.3em}
}






\end{poster}

\end{document}
